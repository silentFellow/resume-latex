%% spaces => \smallskip, \medskip, \bigskip, \vspace

%% Use the "normalphoto" option if you want a normal photo instead of cropped to a circle
% \documentclass[10pt,a4paper,normalphoto]{altacv}

%% AltaCV uses the fontawesome5 and simpleicons packages.
%% See http://texdoc.net/pkg/fontawesome5 and http://texdoc.net/pkg/simpleicons for full list of symbols.


% ...and the markers for the date/location for \cvevent
% \renewcommand{\cvDateMarker}{\faCalendar*[regular]}
% \renewcommand{\cvLocationMarker}{\faMapMarker*}

%% You can add your own arbitrary detail with
%% \printinfo{symbol}{detail}[optional hyperlink prefix]
% \printinfo{\faPaw}{Hey ho!}[https://example.com/]

\documentclass[10pt,a4paper,ragged2e,withhyper]{altacv}

% Change the page layout if you need to
\geometry{left=1.25cm,right=1.25cm,top=1.5cm,bottom=1.5cm,columnsep=1.2cm}

% The paracol package lets you typeset columns of text in parallel
\usepackage{paracol}

\iftutex
  % If using xelatex or lualatex:
  \setmainfont{Roboto Slab}
  \setsansfont{Lato}
  \renewcommand{\familydefault}{\sfdefault}
\else
  % If using pdflatex:
  \usepackage[rm]{roboto}
  \usepackage[defaultsans]{lato}
  % \usepackage{sourcesanspro}
  \renewcommand{\familydefault}{\sfdefault}
\fi

% Change the colours if you want to
\definecolor{SlateGrey}{HTML}{2E2E2E}
\definecolor{LightGrey}{HTML}{666666}
\definecolor{DarkPastelRed}{HTML}{450808}
\definecolor{PastelRed}{HTML}{8F0D0D}
\definecolor{GoldenEarth}{HTML}{E7D192}
\colorlet{name}{black}
\colorlet{tagline}{PastelRed}
\colorlet{heading}{DarkPastelRed}
\colorlet{headingrule}{GoldenEarth}
\colorlet{subheading}{PastelRed}
\colorlet{accent}{PastelRed}
\colorlet{emphasis}{SlateGrey}
\colorlet{body}{LightGrey}

% Change some fonts, if necessary
\renewcommand{\namefont}{\Huge\rmfamily\bfseries}
\renewcommand{\personalinfofont}{\footnotesize}
\renewcommand{\cvsectionfont}{\LARGE\rmfamily\bfseries}
\renewcommand{\cvsubsectionfont}{\large\bfseries}


% Change the bullets for itemize and rating marker
% for \cvskill if you want to
\renewcommand{\cvItemMarker}{{\small\textbullet}}
\renewcommand{\cvRatingMarker}{\faCircle}
% ...and the markers for the date/location for \cvevent
% \renewcommand{\cvDateMarker}{\faCalendar*[regular]}
% \renewcommand{\cvLocationMarker}{\faMapMarker*}


%% Use (and optionally edit if necessary) this .tex if you
%% want to use an author-year reference style like APA(6)
%% for your publication list
% \input{pubs-authoryear.tex}

%% Use (and optionally edit if necessary) this .tex if you
%% want an originally numerical reference style like IEEE
%% for your publication list
\input{pubs-num.tex}

%% sample.bib contains your publications
\addbibresource{sample.bib}
% \usepackage{academicons}\let\faOrcid\aiOrcid
\begin{document}
\name{Gowtham S}
\tagline{Tech Enthusiast | Passionate Developer from India | Aficionado of CLI, Linux, and Open-Source | Crafting Elegant Solutions with Precision ⚡️}
\photo{2.8cm}{profile}

\personalinfo{%
  % Not all of these are required!
  \email{gowtham.sri@zohomail.in}
  \phone{739-58-31324}
  \mailaddress{132, Kavindachipalayam, Erode-638115}
  \location{Tamilnadu, India}
  \homepage{gowtham-portfolio-5idn.onrender.com}
  \linkedin{silentFellow}
  \github{silentFellow}

  \newcommand{\faLeetCode}{\hspace{1mm}\includegraphics[height=1em]{leetcode.png}}
\printinfo{\faLeetCode}{silentFellow}[https://leetcode.com/u/silentFellow]}


\makecvheader
%% Depending on your tastes, you may want to make fonts of itemize environments slightly smaller
% \AtBeginEnvironment{itemize}{\small}

%% Set the left/right column width ratio to 6:4.
\columnratio{0.6}

% Start a 2-column paracol. Both the left and right columns will automatically
% break across pages if things get too long.
\begin{paracol}{2}
\cvsection{Experience}

\cvevent{Freelancing at \href{https://greencollar.ai/}{greencollar.ai}}{}{November 2023 - April 2024}{India}

\cvevent{Internship at greencollar.ai}{Green Collar Agritech uses advanced algorithms and deep domain expertise to deliver real-time crop-based predictions, optimizing operations and ensuring product quality.
}{May 2024 - January 2025}{India}
\begin{itemize}
\item Developed the official website for Green Collar Agritech.
\item Built an internal admin access product for model crop testing and analysis.
\end{itemize}

\cvsection{Projects}

\cvevent{NeoUrl}{A URL shortener}{\httplink{https://neourl.onrender.com}}{}
\begin{itemize}
    \item A simple, yet powerful URL shortener written in go + htmx.
\end{itemize}

\divider

\cvevent{Cred}{CLI tool for managing passwords and environment variables}{\httplink{https://github.com/silentFellow/cred}}{}
\begin{itemize}
    \item Encrypts data using a GPG key for secure storage.
\end{itemize}

\divider

\cvevent{Abstract-UI}{Component Library for React and Next.js}{\httplink{https://www.npmjs.com/package/@silentfellow/abstract-ui}}{}
\begin{itemize}
    \item A library of reusable UI components to speed up web development.
\end{itemize}

\divider

\cvevent{NeoBlog}{A feature-rich blogging platform}{\httplink{https://neoblog.onrender.com}}{}
\begin{itemize}
    \item Built with Next.js, includes authentication and rich text editing.
\end{itemize}

\divider

\medskip

\cvsection{Certifications}

\cvevent{MongoDB Associate Developer}{MongoDB University}{\httplink{https://learn.mongodb.com/c/o4uMvThfQDuYpcjprIEyWg}}{}

\switchcolumn

\cvsection{Strengths}

{\LaTeXraggedright
\cvtag{Curious}
\cvtag{Smart-Working}
\cvtag{Hard-Working}
\cvtag{Leadership}
\cvtag{Teamwork}
\par}

\cvsection{Tech Stack}

\cvskill{Golang}{5}
\divider

\cvskill{JavaScript/TypeScript}{5}
\textit{(React, NextJs, NodeJs)}
\divider

\cvskill{Python}{4}
\divider

\cvskill{C, C++, Java}{3}
\divider

\cvsection{Tech Toolbox}

{\LaTeXraggedright
\cvtag{Linux (Advanced)}
\cvtag{Neovim}
\cvtag{Git}
\cvtag{Docker}
\cvtag{MongoDB}
\cvtag{PostgresSQL}
\cvtag{AWS}
\cvtag{Markdown}
\cvtag{Cred}
\cvtag{Ente-Auth}
\cvtag{Bruno}
\par}

\medskip

\cvsection{Education}

\cvevent{B.Tech AIML - 8.84 CGPA}{Kongu Engineering College}{Nov 2022 -- Ongoing}{}

\divider

\cvevent{HSC - 92.03\%}{Navarasam Matric. Hr. Sec. School}{May 2021}{}

\divider

\cvevent{SSLC - 89.80\%}{Navarasam Matric. Hr. Sec. School}{Mar 2019}{}
89.08\%

\divider

\cvsection{Miscellaneous}

\begin{itemize}
    \item 2024: 30-Hour Hackathon Winner
    \item 2023: Published Paper on Enhanced Weather Prediction using ML \href{https://github.com/silentFellow/weatherPrediction.git}{(Link)}
    \item 2023: AI Association Member in College Department
\end{itemize}


\end{paracol}

\end{document}

% Change the font if you want to, depending on whether
% you're using pdflatex or xelatex/lualatex
% WHEN COMPILING WITH XELATEX PLEASE USE
% xelatex -shell-escape -output-driver="xdvipdfmx -z 0" sample.tex

%% For services and platforms like Mastodon where there isn't a
%% straightforward relation between the user ID/nickname and the hyperlink,
%% you can use \printinfo directly e.g.
% \printinfo{\faMastodon}{@username@instace}[https://instance.url/@username]
%% But if you absolutely want to create new dedicated info fields for
%% such platforms, then use \NewInfoField* with a star:
% \NewInfoField*{mastodon}{\faMastodon}
%% then you can use \mastodon, with TWO arguments where the 2nd argument is
%% the full hyperlink.
% \mastodon{@username@instance}{https://instance.url/@username}
