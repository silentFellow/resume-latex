%% spaces => \smallskip, \medskip, \bigskip, \vspace

%% Use the "normalphoto" option if you want a normal photo instead of cropped to a circle
% \documentclass[10pt,a4paper,normalphoto]{altacv}

%% AltaCV uses the fontawesome5 and simpleicons packages.
%% See http://texdoc.net/pkg/fontawesome5 and http://texdoc.net/pkg/simpleicons for full list of symbols.


% ...and the markers for the date/location for \cvevent
% \renewcommand{\cvDateMarker}{\faCalendar*[regular]}
% \renewcommand{\cvLocationMarker}{\faMapMarker*}

%% You can add your own arbitrary detail with
%% \printinfo{symbol}{detail}[optional hyperlink prefix]
% \printinfo{\faPaw}{Hey ho!}[https://example.com/]

\documentclass[10pt,a4paper,ragged2e,withhyper]{altacv}

% Change the page layout if you need to
\geometry{left=1.25cm,right=1.25cm,top=1.5cm,bottom=1.5cm,columnsep=1.2cm}

% The paracol package lets you typeset columns of text in parallel
\usepackage{paracol}

\iftutex
  % If using xelatex or lualatex:
  \setmainfont{Roboto Slab}
  \setsansfont{Lato}
  \renewcommand{\familydefault}{\sfdefault}
\else
  % If using pdflatex:
  \usepackage[rm]{roboto}
  \usepackage[defaultsans]{lato}
  % \usepackage{sourcesanspro}
  \renewcommand{\familydefault}{\sfdefault}
\fi

% Change the colours if you want to
\definecolor{SlateGrey}{HTML}{2E2E2E}
\definecolor{LightGrey}{HTML}{666666}
\definecolor{DarkPastelBlack}{HTML}{181818} % DarkPastelRed 450808
\definecolor{PastelBlack}{HTML}{292929} % PastelRed 8F0D0D
\definecolor{GoldenEarth}{HTML}{E7D192}
\colorlet{name}{black}
\colorlet{tagline}{PastelBlack}
\colorlet{heading}{DarkPastelBlack}
\colorlet{headingrule}{GoldenEarth}
\colorlet{subheading}{PastelBlack}
\colorlet{accent}{PastelBlack}
\colorlet{emphasis}{SlateGrey}
\colorlet{body}{LightGrey}

% Change some fonts, if necessary
\renewcommand{\namefont}{\Huge\rmfamily\bfseries}
\renewcommand{\personalinfofont}{\footnotesize}
\renewcommand{\cvsectionfont}{\Large\rmfamily\bfseries}
\renewcommand{\cvsubsectionfont}{\large\bfseries}

% Change the bullets for itemize and rating marker
% for \cvskill if you want to
\renewcommand{\cvItemMarker}{{\small\textbullet}}
\renewcommand{\cvRatingMarker}{\faCircle}
% ...and the markers for the date/location for \cvevent
% \renewcommand{\cvDateMarker}{\faCalendar*[regular]}
% \renewcommand{\cvLocationMarker}{\faMapMarker*}


%% Use (and optionally edit if necessary) this .tex if you
%% want to use an author-year reference style like APA(6)
%% for your publication list
% \input{pubs-authoryear.tex}

%% Use (and optionally edit if necessary) this .tex if you
%% want an originally numerical reference style like IEEE
%% for your publication list
\input{pubs-num.tex}

%% sample.bib contains your publications
\addbibresource{sample.bib}
% \usepackage{academicons}\let\faOrcid\aiOrcid
\begin{document}
\name{Gowtham S}
\tagline{Tech Enthusiast | Passionate about Linux, CLI and Open-Source | Curious Mind with a Professional Approach to Problem-Solving and Innovation ⚡️}
\photo{2.8cm}{profile}

\personalinfo{%
  % Not all of these are required!
  \email{gowtham.sri@zohomail.in}
  \phone{+91 739-583-1324}
  \location{Tamilnadu, India}


  \\ \vspace{1em}
  \newcommand{\faLeetCode}{\hspace{1mm}\includegraphics[height=1em]{leetcode.png}}
  \href{https://github.com/silentFellow}{\faGithub} \hspace{0.5em} | \hspace{0.5em}
  \href{https://www.linkedin.com/in/silentFellow}{\faLinkedin} \hspace{0.5em} | \hspace{0.5em}
  \href{https://gowthams-portfolio.vercel.app}{\faGlobe} \hspace{0.5em} | \hspace{0.5em}
  \href{https://leetcode.com/u/silentFellow}{\faLeetCode}
  \par
}

\makecvheader
%% Depending on your tastes, you may want to make fonts of itemize environments slightly smaller
% \AtBeginEnvironment{itemize}{\small}

%% Set the left/right column width ratio to 6:4.
\columnratio{0.6}

% Start a 2-column paracol. Both the left and right columns will automatically
% break across pages if things get too long.
\begin{paracol}{2}

% if more things, then use multicol
% \usepackage{multicol}
% \vspace{-4em}
% \begin{multicols}{2}
% \begin{itemize}
%     \item ...
% \end{itemize}
% \end{multicols}
\cvsection{Area of Interest}
\begin{itemize}
    \item Full-Stack Development
    \item Deep Learning
    \item Linux \& CLI
    \item Open Source Contribution
\end{itemize}

\cvsection{Experience}

\cvevent{Freelancing at \href{https://greencollar.ai/}{greencollar.ai}}{}{November 2023 - April 2024}{India}

\cvevent{Internship at \href{https://greencollar.ai/}{greencollar.ai}}{Green Collar Agritech uses advanced algorithms and deep domain expertise to deliver real-time crop-based predictions, optimizing operations and ensuring product quality.
}{May 2024 - January 2025}{India}
\begin{itemize}
\item Developed the official website for Green Collar Agritech.
\item Built an internal admin access product for model crop testing and analysis.
\end{itemize}

\cvsection{Projects}

\cvproject
  {NeoUrl}
  {https://github.com/silentFellow/neourl.git}
  {http://43.204.114.237:8080}
  {}
  {A URL shortener written in Go and htmx. It allows users to shorten URLs effectively, providing a simple and efficient way to create shorter links for easier sharing.}
  {Go, htmx}

\divider

\cvproject
  {Cred}
  {https://github.com/silentFellow/cred.git}
  {https://silentfellow.github.io/cred/book}
  {}
  {A CLI tool built for managing passwords and environment variables. It uses GPG for secure data encryption and store data in Git repository.}
  {Go, Cobra Cli}

\divider

\cvproject
  {Abstract-UI}
  {https://github.com/silentFellow/abstract-ui.git}
  {https://abstract-ui-olive.vercel.app}
  {https://www.npmjs.com/package/@silentfellow/abstract-ui}
  {A component library for React and Next.js. It offers reusable UI components to speed up web development.}
  {React, TypeScript, Tailwind CSS, Storybook}

\divider

\cvproject
  {NeoBlog}
  {https://github.com/silentFellow/neoblog.git}
  {https://neoblog-ruddy.vercel.app}
  {}
  {A blogging platform built with Next.js. Features rich text editing, multimedia support, and SEO optimization.}
  {NextJs, React, TypeScript, Tailwind CSS, Shadcn-UI, PostgresSQL}

\medskip

\cvsection{Certifications}

\cvcertification{MongoDB Associate Developer}{https://learn.mongodb.com/c/o4uMvThfQDuYpcjprIEyWg}

\switchcolumn

\cvsection{Strengths}

{\LaTeXraggedright
\cvtag{Curious}
\cvtag{Smart-Working}
\cvtag{Leadership}
\cvtag{Teamwork and Collaboration}
\cvtag{Creativity}
\cvtag{Lateral Thinking}
\cvtag{Quick Learner}
\cvtag{Empathy}
\cvtag{Strong Intuition}
\par}

\cvsection{Tech Stack}

\cvskill{Golang}{5}
\divider

\cvskill{JavaScript/TypeScript}{5}
\textit{(React, NextJs, NodeJs)}
\divider

\cvskill{Python}{4}
\divider

\cvskill{C, C++, Java}{3}
\divider

\cvsection{Tech Toolbox}

{\LaTeXraggedright
\cvtag{Linux (Advanced)}
\cvtag{Neovim}
\cvtag{Git}
\cvtag{Docker}
\cvtag{MongoDB}
\cvtag{PostgresSQL}
\cvtag{AWS}
\cvtag{Markdown}
\cvtag{Cred}
\cvtag{Ente-Auth}
\cvtag{Bruno}
\cvtag{Postman}
\par}

\medskip

\cvsection{Education}

\cvevent{B.Tech AIML - 8.84 CGPA}{Kongu Engineering College}{Nov 2022 -- Ongoing}{}

\divider

\cvevent{HSC - 92.03\%}{Navarasam Matric. Hr. Sec. School}{May 2021}{}

\divider

\cvevent{SSLC - 89.80\%}{Navarasam Matric. Hr. Sec. School}{Mar 2019}{}

\divider

\cvsection{Achievements}

\begin{itemize}
    \item 2024: 30-Hour Hackathon 2nd Prize
    \item 2023: Published Paper on Enhanced Weather Prediction using ML \href{https://github.com/silentFellow/weatherPrediction.git}{(Link)}
    \item 2023: AI Association Member in College Department
\end{itemize}


\end{paracol}

\end{document}

% Change the font if you want to, depending on whether
% you're using pdflatex or xelatex/lualatex
% WHEN COMPILING WITH XELATEX PLEASE USE
% xelatex -shell-escape -output-driver="xdvipdfmx -z 0" sample.tex

%% For services and platforms like Mastodon where there isn't a
%% straightforward relation between the user ID/nickname and the hyperlink,
%% you can use \printinfo directly e.g.
% \printinfo{\faMastodon}{@username@instace}[https://instance.url/@username]
%% But if you absolutely want to create new dedicated info fields for
%% such platforms, then use \NewInfoField* with a star:
% \NewInfoField*{mastodon}{\faMastodon}
%% then you can use \mastodon, with TWO arguments where the 2nd argument is
%% the full hyperlink.
% \mastodon{@username@instance}{https://instance.url/@username}
